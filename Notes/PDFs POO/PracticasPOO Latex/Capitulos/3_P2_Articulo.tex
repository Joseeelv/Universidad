La clase Artículo contiene 5 atributos → código de referencia, título, fecha de publicación, 
precio y número de ejemplares a la venta. Los 3 primeros son no modificables.

Un objeto de la clase Artículo se construye mediante esos 5 atributos en el orden: 
referencia, título, fecha de publicación, precio y existencias.
Este último parámetro es opcional: si no se suministra, se toma como cero.

La clase Artículo tendrá los métodos observadores: \texttt{referencia()}, \texttt{titulo()},
 \texttt{f\_publi()},\\ \texttt{precio()} y \texttt{stock()}. 
 Estos dos últimos estarán sobrecargados para devolver una referencia al atributo correspondiente, con el fin de permitir su modificación.

Finalmente, contará con un operador de inserción en flujo (\texttt{<<}) que imprimirá los 
datos de un artículo con el formato:

\begin{center}
\texttt{"Fundamentos de C++", 1998. 29,95 €}
\end{center}

\subsection{Articulo.hpp}
\begin{minted}[breaklines]{C++}
#ifndef ARTICULO_HPP
#define ARTICULO_HPP

//Inclusión de librerías
#include "../P1/fecha.hpp"
#include "../P1/cadena.hpp"
#include <iostream>
#include <iomanip>
#include <locale.h>
class Articulo{
public:
    Articulo(Cadena referencia, Cadena titulo ,Fecha f_publi
    ,double precio,unsigned ejemplares =0):referencia_(referencia),titulo_(titulo),
    fpubli_(f_publi),precio_(precio),ejemplares_(ejemplares){}

    //Observadores de la clase
    inline const Cadena& referencia() const noexcept{return referencia_;}
    inline const Cadena& titulo()const noexcept{return titulo_;}
    inline const Fecha& f_publi()const noexcept{return fpubli_;}
    inline double precio()const noexcept{return precio_;}
    inline double& precio()noexcept{return precio_;}
    inline unsigned stock()const noexcept{return ejemplares_;}
    inline unsigned& stock()noexcept{return ejemplares_;}

private:
    const Cadena referencia_,titulo_;
    const Fecha fpubli_;
    double precio_;
    unsigned ejemplares_;
};

//Operador de inserción en flujo
std::ostream& operator <<(std::ostream& , const Articulo&)noexcept;
#endif // !ARTICULO_HPP 
\end{minted}
\newpage
\subsection{Articulo.cpp}
\begin{minted}[breaklines]{C++}
#include "articulo.hpp"

//Operador de inserción en flujo
std::ostream& operator <<(std::ostream& output, const Articulo& art)noexcept{
std::locale::global(std::locale(""));
output<<"["<<art.referencia()<<"] "<<"\""<<art.titulo()<<"\""<<", "
        <<art.f_publi().anno()<<". "
        <<std::fixed<<std::setprecision(2)<<art.precio()<<" €";
return output;
}
\end{minted}