\thispagestyle{empty}
\chapter*{Introducción a las Prácticas de POO}

En este PDF vas a encontrar las prácticas de la asignatura de Programación Orientada a Objetos
del curso 2023 - 2024.
Contiene una explicación detallada de lo que se va a hacer en cada una de las prácticas así
como el código ya implementado.

Constará de 5 prácticas donde las 2 primeras las (prácticas 0 y 1) se centrarán
en la definición e implementación de dos clases para poder aprender así los 
principios de la Programación Orientada a Objetos, es decir, el primer parcial.

Las prácticas restantes (prácticas 2 y 3) vamos a aprender a como relacionar clases
entre sí teniendo como referencia algunos diagramas de clases que nos proporcionará
el enunciado de la misma.

En la práctica 4 tendremos que hacer uso de polimorismo y de mecanismos de conversiones explicitas para poder convertir punteros de la clase Base a clases Derivadas.

Cada fichero de cabecera irá en un \texttt{.hpp} y la implementación de los métodos de 
cada fichero de cabecera irá en un \texttt{.cpp}.

Para comprobar que las prácticas están resueltas correctamente se hará uso de un
entorno de pruebas en \textit{Docker}, en el Campus Virtual se encuentra toda la información
para su correcta instalación.