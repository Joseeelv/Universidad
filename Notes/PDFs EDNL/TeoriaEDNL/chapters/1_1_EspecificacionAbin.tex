Sea la especificación de operaciones del TAD Abin (árbol binario):
\subsection*{Constructor del árbol binario}
  \underbar{\textit{Postcondición:}} Crea un árbol completamente vacío.\\
  \verb|  Abin();|

\subsection*{Inserción de nodos}
\begin{itemize}
  \item \underbar{\large\textbf{Inserción del nodo raíz}}:\\
  \underbar{\textit{Precondición:}} Árbol vacío.\\
  \underbar{\textit{Postcondición:}} Inserta el nodo raíz cuyo contenido es `e'.\\
  \verb|  void insertarRaiz(const T& e);|
  \item \underbar{\large\textbf{Inserción de los nodos hijos}}:\\
  \underbar{\textit{Precondición:}} n es un nodo que existe en el árbol y no tiene hijo izquierdo/derecho.\\
  \underbar{\textit{Postcondición:}} Inserta el nodo hijo(izquierdo o derecho) cuyo contenido es `e'.\\
  \verb|  void insertarHijoIzqdo(nodo n, const T& e);|\\
  \verb|  void insertarHijoDrcho(nodo n, const T& e);|
\end{itemize}

\subsection*{Eliminación de nodos}
Para poder eliminar un nodo éste debe de ser un \textbf{nodo hoja}, es decir, no tiene descendientes. Si no hacemos esto, no estaríamos cumpliendo la especificación del TAD.

Además no nos podemos eliminar a nosotros mismo, por tanto, esta operación solamente se podrá realizar a los hijos de nodo, si ese nodo no tiene hijos(es hoja) para eliminarlo tendremos que llamar a su padre y eliminarlo.

\begin{itemize}
  \item \underbar{\large\textbf{Eliminación del nodo raíz}}:\\
  \underbar{\textit{Precondición:}} Árbol no vacío y el nodo raiz es una hoja.\\
  \underbar{\textit{Postcondición:}} Elimina el nodo raíz y deja el árbol vacío.\\
  \verb|  void eliminarRaiz();|

  \item \underbar{\large\textbf{Eliminación de los nodos hijos}}:\\
  \underbar{\textit{Precondición:}} Árbol no vacío y el nodo n no es una hoja.\\
  \underbar{\textit{Postcondición:}} Elimina el nodo hijo izquierdo o derecho del nodo n.\\
  \verb| void eliminarHijoIzqdo(nodo n);|\\
  \verb| void eliminarHijoDrcho(nodo n);|
\end{itemize}

\subsection*{Métodos observadores}
\begin{itemize}
  \item \underbar{\large\textbf{Árbol vacío}}:\\
  \underbar{\textit{Postcondición:}} Devuelve \texttt{TRUE} si el árbol está vacío, si no devuelve \texttt{FALSE}.\\
  \verb|  bool arbolVacio();|
  \item \underbar{\large\textbf{Obtener el elemento de un nodo}}:\\
  \underbar{\textit{Precondición:}} n es un nodo que existe en el árbol. \\
  \underbar{\textit{Postcondición:}} Devuelve el elemento de ese nodo.\\
  \verb|  const T& elemento(nodo n) const;|\\
  \verb|  T& elemento(nodo n);|
  \item \underbar{\large\textbf{Obtener el elemento de la raíz}}:\\
  \underbar{\textit{Postcondición:}} Devuelve el nodo raíz, si el árbol está vacío devuelve \texttt{NODO\_NULO}.\\
  \verb| nodo raiz()const;|
  \item \underbar{\large\textbf{Obtener nodo padre de un nodo cualquiera}}:\\
  \underbar{\textit{Precondición:}} n es un nodo que existe en el árbol.\\
  \underbar{\textit{Postcondición:}} Devuelve el padre del nodo n, si este no tiene devuelve \texttt{NODO\_NULO}.\\
  \verb|  nodo padre(nodo n)const;|
  \item \underbar{\large\textbf{Obtener Hijo Izquierdo de un nodo}}:\\
  \underbar{\textit{Precondición:}} n es un nodo que existe en el árbol.\\
  \underbar{\textit{Postcondición:}} Devuelve el hijo izquierdo del nodo n, si este no tiene devuelve \texttt{NODO\_NULO}.\\
  \verb|  nodo hijoIzqdo(nodo n)const;| 
  \item \underbar{\large\textbf{Obtener Hijo Derecho de un nodo}}:\\
  \underbar{\textit{Precondición:}} n es un nodo que existe en el árbol.\\
  \underbar{\textit{Postcondición:}} Devuelve el hijo derecho del nodo n, si este no tiene devuelve \texttt{NODO\_NULO}.\\
  \verb|  nodo hijoDrcho(nodo n)const;| 
\end{itemize}

\textit{NOTA:} \texttt{NODO\_NULO} no es un elemento, si no que indica la no existencia de un nodo.
