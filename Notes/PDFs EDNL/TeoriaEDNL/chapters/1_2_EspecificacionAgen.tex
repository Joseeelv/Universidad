Sea la especificación de las operaciones del TAD árbol general (Agen):

\subsection*{Constructor del árbol general}
\underbar{\textit{Postcondición:}} Crea y devuelve un árbol general vacío.\\
\verb|  Agen();|

\subsection*{Inserción de nodos}
\begin{itemize}
  \item \underbar{\large\textbf{Inserción del nodo raíz}}:\\
  \underbar{\textit{Precondición:}} Árbol vacío.\\
  \underbar{\textit{Postcondición:}} Inserta el nodo raíz cuyo contenido es `e'.\\
  \verb|  void insertarRaiz(const T& e);|

  \item \underbar{\large\textbf{Inserción de los nodos hijos}}:\\
  \underbar{\textit{Precondición:}} n es un nodo que existe en el árbol.\\
  \underbar{\textit{Postcondición:}} Inserta el elemento `e' como hijo izquierdo de n, si ya tiene un hijo izquierdo el nodo n, este lo inserta y el anterior pasa a ser hermano derecho del nuevo.\\
  \verb|  void insertarHijoIzqdo(nodo n, const T& e);|

  \item \underbar{\large\textbf{Inserción de los nodos hermanos}}:\\
  \underbar{\textit{Precondición:}} n es un nodo que existe en el árbol y no es raíz.\\
  \underbar{\textit{Postcondición:}} Inserta el elemento `e' como hermano derecho del nodo n.\\
  \verb|  void insertarHermDrcho(nodo n,const T& e);|
\end{itemize}
\subsection*{Eliminación de nodos}
Al igual que en los árboles binarios, tenemos que comprobar que el nodo que vamos a eliminar sea una hoja, ya sea un hijo izquierdo o un hermano derecho de un nodo cualquiera.
\begin{itemize}
  \item \underbar{\large\textbf{Eliminación del nodo raíz}}:\\
  \underbar{\textit{Precondición:}} Árbol no vacío y nodo raíz es una hoja.\\
  \underbar{\textit{Postcondición:}} Elimina el nodo raíz y deja el árbol vacío.\\
  \verb|  void eliminarRaiz();|
  \item \underbar{\large\textbf{Eliminación de nodo hijo izquierdo}}:\\
  \underbar{\textit{Precondición:}} n es un nodo del árbol, tiene hijo izquierdo y este es Hoja.\\
  \underbar{\textit{Postcondición:}} Elimina el hijo izquierdo del nodo n y su hermano derecho pasa a ser el nuevo hijo izquierdo.\\
  \verb|  void eliminarHijoIzqdo(nodo n);|
  \item \underbar{\large\textbf{Eliminación de nodo hermano derecho}}:\\
  \underbar{\textit{Precondición:}} n es un nodo del árbol, tiene hermano derecho de n y este es Hoja.\\
  \underbar{\textit{Postcondición:}} Elimina el hermano derecho de nodo n.\\
  \verb|  void eliminarHermDrcho(nodo n);|
\end{itemize}
\subsection*{Métodos observadores}
\begin{itemize}
  \item \underbar{\large\textbf{Consultar elemento de un nodo}}:\\
  \underbar{\textit{Precondición:}} n es un nodo del árbol.\\
  \underbar{\textit{Postcondición:}} Devuelve el contenido del nodo n.\\
  \verb|  const T& elemento(nodo n)const;|\\
  \verb|  T& elemento(nodo n);|
  \item \underbar{\large\textbf{Consultar la raíz del árbol}}:\\
  \underbar{\textit{Postcondición:}} Devuelve el contenido del nodo raíz, no tiene, devuelve NODO\_NULO.\\
  \verb|  nodo raiz()const;|
  \item \underbar{\large\textbf{Consultar el padre de un nodo}}:\\
  \underbar{\textit{Precondición:}} n es un nodo del árbol.\\
  \underbar{\textit{Postcondición:}} Devuelve el padre del nodo n, si no tiene, devuelve NODO\_NULO.\\
  \verb|  nodo padre(nodo n)const;|
  \item \underbar{\large\textbf{Consultar el hijo izquierdo de un nodo}}:\\
  \underbar{\textit{Precondición:}} n es un nodo del árbol.\\
  \underbar{\textit{Postcondición:}} Devuelve el hijo izqdo del nodo n, si no tiene, devuelve NODO\_NULO.\\
  \verb|  nodo hijoIzqdo(nodo n)const;|
  \item \underbar{\large\textbf{Consultar el hermano derecho de un nodo}}:\\
  \underbar{\textit{Precondición:}} n es un nodo del árbol.\\
  \underbar{\textit{Postcondición:}} Devuelve el hermano derecho del nodo n, si no existe, devuelve\\NODO\_NULO.\\
  \verb|  nodo hermDrcho(nodo n)const;|
\end{itemize}