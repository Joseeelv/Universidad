\textbf{\underbar{Enunciado:}}\textit{ Se necesita hacer un estudio de las distancias mínimas necesarias para viajar entre dos ciudades cualesquiera de un país llamado Zuelandia.}

\textit{El problema es sencillo pero hay que tener en cuenta unos pequeños detalles:
\begin{enumerate}[label = \alph*)]
  \item La orografía de Zuelandia es un poco especial, las carreteras son muy estrechas y por tanto solo permiten un sentido de la circulación.
  \item Actualmente Zuelandia es un país en guerra. Y de hecho hay una serie de ciudades del país que han sido tomadas por los rebeldes, por lo que no pueden ser usadas para viajar.
  \item Los rebeldes no sólo se han apoderado de ciertas ciudades del país, sino que también han cortado ciertas carreteras, (por lo que estas carreteras no pueden ser usadas).
  \item Pero el gobierno no puede permanecer impasible ante la situación y ha exigido que absolutamente todos los viajes que se hagan por el país pasen por la capital del mismo, donde se harán los controles de seguridad pertinentes.
\end{enumerate}
Dadas estas cuatro condiciones, se pide implementar un subprograma que dados el grafo (matriz de costes) de Zuelandia en situación normal, la relación de las ciudades tomadas por los rebeldes, la relación de las carreteras cortadas por los rebeldes y la capital de Zuelandia, calcule la matriz de costes mínimos para viajar entre cualesquiera dos ciudades zuelandesas en esta situación.
}





\begin{minted}[breaklines]{C++}
  
\end{minted}


Hemos hecho uso de las operaciones de los grafos:
\begin{itemize}
  \item \verb | |
  \begin{itemize}
    \item \textit{Pre}:
    \item \textit{Post}:
  \end{itemize}
\end{itemize}