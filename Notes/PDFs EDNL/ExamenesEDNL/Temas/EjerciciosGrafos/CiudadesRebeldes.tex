\textbf{\underbar{Enunciado:}}\textit{ Se necesita hacer un estudio de las distancias mínimas necesarias para viajar entre dos ciudades cualesquiera de un país llamado Zuelandia.}

\textit{El problema es sencillo pero hay que tener en cuenta unos pequeños detalles:
\begin{enumerate}[label = \alph*)]
  \item La orografía de Zuelandia es un poco especial, las carreteras son muy estrechas y por tanto solo permiten un sentido de la circulación.
  \item Actualmente Zuelandia es un país en guerra. Y de hecho hay una serie de ciudades del país que han sido tomadas por los rebeldes, por lo que no pueden ser usadas para viajar.
  \item Los rebeldes no sólo se han apoderado de ciertas ciudades del país, sino que también han cortado ciertas carreteras, (por lo que estas carreteras no pueden ser usadas).
  \item Pero el gobierno no puede permanecer impasible ante la situación y ha exigido que absolutamente todos los viajes que se hagan por el país pasen por la capital del mismo, donde se harán los controles de seguridad pertinentes.
\end{enumerate}
Dadas estas cuatro condiciones, se pide implementar un subprograma que dados el grafo (matriz de costes) de Zuelandia en situación normal, la relación de las ciudades tomadas por los rebeldes, la relación de las carreteras cortadas por los rebeldes y la capital de Zuelandia, calcule la matriz de costes mínimos para viajar entre cualesquiera dos ciudades zuelandesas en esta situación.
}

Partimos de la base de que tenemos un grafo ponderado y dirigido (ya que solo se puede ir en una dirección), contamos con el conjunto de ciudades y carreteras (unión de dos ciudades) que están tomadas por los rebeldes, a las cuales no podremos acceder. Además tenemos la capital de Zuelandia, que nos servirá para hacer que todos los viajes se realicen por la misma.

Como queremos obtener las distancias mínimas de viajar entre dos ciudades cualesquiera de Zuelandia, haremos uso del algoritmo de Floyd, ya que este calcula los costes de viajar entre cada par de vértices de todo el grafo.

\newpage
\begin{minted}[breaklines]{C++}
//Definimos los tipos de datos a usar, representaremos una ciudad como un entero sin signo.
typedef std::pair<size_t,size_t> Carretera;
typedef matriz<size_t>CostesViajes;

CostesViajes ZuelandiaRebeldes(GrafoP<size_t> &G, const vector<size_t> &CiudadesRebeldes, const vector<Carretera> &CarreterasRebeldes, size_t Capital){
  //Como tenemos ciudades que no pueden ser accesibles, vamos a modificar el grafo, haciendo que el coste de estas sea infinito.
  for(auto &ciudadrebelde : CiudadesRebeldes)
    for(size_t i = 0; i < G.numVert(); i++)
      G[i][ciudadrebelde] = GrafoP<size_t>::INFINITO;

  //Ahora vamos a hacer inaccesibles dichas carreteras
  for(auto & carreterarebelde : CarreterasRebeldes)
    G[carreterarebelde.first][carreterarebelde.second] = GrafoP<size_t>::INFINITO;

  //Ahora vamos a obligar que todos los viajes se lleven a cabo por la Capital
  for(size_t i = 0; i < G.numVert(); i++)
    if(i != Capital){
      for(size_t j = i+1; j <G.numVert(); j++){
        if(j!= Capital)
        G[i][j] = GrafoP<size_t>::INFINITO;
      }
    }
  //Ya teniendo tanto las ciuades rebeldes como la carreteras rebeldes indicadas en el grafo, y todos los viajes cruzan la capital, vamos a poder calcular los costes mínimos de los caminos entre cada par de vértices del grafo,es decir floyd, para ello, creamos las matrices de costes mínimos y vértices.
  matriz<size_t>Vertices(G.numVert()),CosteMinimos(G.numVert());
  return CosteMinimos = Floyd(G,Vertices); 
}
\end{minted}

Hemos hecho uso de las operaciones de los grafos y prototipos de las funciones:
\begin{itemize}
  \item \verb |size_t numVert() const;|
  \begin{itemize}
    \item \textit{Post}: Devuelve el número de vértices del grafo.
  \end{itemize}
  \item \verb |matriz<size_t>(G.numVert());|
  \begin{itemize}
    \item \textit{Post}: Crea y devuelve una matriz de enteros sin signos vacía de tamaño G.numVert().
  \end{itemize}
  \item \verb |vector<tCoste>& operator[](vertice v)const;|
  \begin{itemize}
    \item \textit{Post}: Devuelve un vector con los costes de las aristas adyacentes a v.
  \end{itemize}
  \item \verb |const T& GrafoP<tCoste>::INFINITO = std::numeric_limits<T>::max();|
  \begin{itemize}
    \item \textit{Post}: Devuelve el valor máximo permitido.
  \end{itemize}
  \item \verb |matriz<tCoste> Floyd(const GrafoP<size_t> &G,|\\ 
  \hspace*{0.5cm} \verb|vector<typename GrafoP<tCoste>::vertice> &P);|
  \begin{itemize}
    \item \textit{Pre}: Recibe un Grafo ponderado y una matriz de vértices.
    \item \textit{Post}: Devuelve una matriz con los costes mínimos y una matriz con los vértices por los que pasan los caminos.
  \end{itemize}
\end{itemize}