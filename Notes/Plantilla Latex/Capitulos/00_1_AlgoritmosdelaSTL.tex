\thispagestyle{empty}
\cleardoublepage
\chapter{Teoría de la STL}

La \textit{STL} (Standard Template Library) de C++ es una parte integral del estándar de C++ que proporciona una colección de plantillas y algoritmos genéricos, así como tipos de datos contenedores y funciones asociadas para facilitar el desarrollo de software.

La \textit{STL} fue diseñada para ser flexible, eficiente y fácil de usar.

La \textit{STL} proporciona una amplia gama de algoritmos genéricos que operan con los tipos de datos contenedores como: vectores \texttt{vector}, listas doblemente enlazadas \texttt{list},
bicolas \texttt{deque}, pilas \texttt{stack} o contenedores que contienen una colección de objetos de un tipo determinado.

En este último caso encontramos \texttt{set} (conjunto de objetos de un tipo determinado) y \texttt{map} (diccionario clave - valor).
Además encontramos sus variantes donde podemos encontrar valores repetidos como son \texttt{multiset} ó \texttt{multimap} y sus variantes donde el contenido de estos contenedores no están ordenados \texttt{unordered\_set} y \texttt{unordered\_map}, respectivamente.

\section{Iteradores}
Mediante los \texttt{iterators} (iteradores) podemos acceder a las posiciones o elementos contenidos en cada tipo de contenedor.

Se recomienda el uso de la \textit{keyword} \texttt{auto}, para que el compilador le asigne el tipo adecuado al iterador.
\section{Algoritmos de búsqueda y ordenación}
