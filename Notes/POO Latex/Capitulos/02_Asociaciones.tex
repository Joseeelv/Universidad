\chapter{Asociaciones entre clases}


Se representa mediante una línea continua entre las clases asociadas.
En este caso podemos saber que dado un libro podemos saber sus autores o viceversa.
Las asociaciones entre clases contienen 3 factores (cardinalidad, navegabilidad y multiplicidad).
\begin{itemize}
	\item La \textbf{cardinalidad} es el número de clases asociadas, por defecto, son binarias (cardinalidad = 2).
	\item La \textbf{multiplicidad} indica cuántos objetos de las clases se enlazan, en este caso como mínimo se enlaza 1 y como máximo muchos (1..*). Si no se especifica es 1.
	\item La \textbf{navegabilidad} indica el sentido de la relación, por defecto, son bidireccionales pero pueden existir asociaciones de \textit{A → B} o \textit{A ← B} (unidireccionales, donde el extremo de la flecha es el sentido).
\end{itemize}
Estas asociaciones pueden tener algunas dependencias (agregaciones y composiciones).
\section{Implementaciones}
Dependiendo de la multiplicidad que haya en cada extremo de las clases de objetos implemetaremos de una manera u 
otra las relaciones entre las clases.

Donde las relaciones que tienen una multiplicidad muchos - muchos se harán uso de tipos contendores asociativos o si la 
multiplicidad es 1 - 1 solamente se almacenará el objeto con el que se enlazan.

\subsection{Plantilla Implementación 1-N}
\begin{center}
\begin{lstlisting}[frame=single]
class A{ //clase con multiplicidad 1
 public:
  //...
  void setB(B&); //enlazamos
  B& getA() const; //objeto B enlazado con A
 private:
  B* b_; //enlace con objeto B
};
class B{ //clase con multiplicidad N
 public:
  //...
  void setA(A&);
  A getB()const;
 private:
  A a_;
};
\end{lstlisting}
\end{center}
\newpage
\subsection{Implementación 1-1}

\begin{figure}[h]
    \centering
	\includegraphics[width=\textwidth]{Imagenes/asociacion1.png}
    \caption{Asociación 1-1}
\end{figure}
\begin{center}
	\begin{lstlisting}[frame=single]
class Persona{
 public:
  //...
  void setB(Asignatura&); //enlazamos   
  //objeto Asignatura enlazado con Persona
  Asignatura& getA() const; 
 private:
  Asignatura* b_; //enlace con objeto Asignatura
};
class Asignatura{
 public:
  //...
  void setA(Persona&); //enlazamos
  Persona& getA()const; //objeto Persona enlazado con Asignatura
 private:
  Persona* a_; //enlace con objeto Persona
};

//Implementacion de los metodos, pueden ser inline
void Persona::setB(Asignatura& b){ b_ = &b;}
Persona::Asignatura& Persona::getB()const{ return *b_;}

//Analogo para obtener Persona
\end{lstlisting}
\end{center}


\newpage
\subsection{Implementación N-M}
\begin{figure}[h]
    \centering
    \includegraphics[width=\textwidth]{Imagenes/asociacion2.png}
    \caption{Asociación muchos - muchos}
\end{figure}

\begin{center}
	\begin{lstlisting}[frame=single]
class Persona{
 public:
  //...
  typedef set<Asignatura*> Bs;
  void setB(Asignatura&);
  const Bs& getA()const;
 private:
  Bs bs_; 
};

class Asignatura{
  public:
    //...
    typedef set<Persona*> As;
    void setA(Persona&);
    const As& getB()const;
  private:
    As as_;
};

//Se pueden hacer inline en la definicion de la clase
void Persona::setB(Asignatura& b){ as_.insert(&b); }
const Persona::Bs& getA()const{ return bs_; }

//Analogo para obtener Persona
\end{lstlisting}
\end{center}

