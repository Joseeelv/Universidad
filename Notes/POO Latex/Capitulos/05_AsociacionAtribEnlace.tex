\chapter{Asociación con atributo de Enlace}
Es una relación de asociación donde encontramos un atributo que se enlaza con la relación de las clases A - B.
Ese atributo no pertenece ni a la clase A ni a la clase B, si no a la relación.
\begin{figure}[h]
	\centering
	\includegraphics[width=\textwidth]{Imagenes/atribenlace.png}
	\caption{Asociación con atributo de enlace}
\end{figure}

Vemos que una persona imparte asignaturas un número de horas y las asignaturas son impartidas por personas un número de horas.

Dependiendo de la multiplicidad podemos hacer que dicho atributo pertenezca a una de las dos clases (la que tiene mayor multiplicidad).

Si Asignatura tuviera multiplicidad 1, el atributos \textit{horas} iría en la clase Persona → hacemos uso de un conjunto ‘\texttt{set}’.

En este caso, ambos tienen multiplicidad muchos - muchos, por tanto, ese atributo no puede almacenarse en ninguna de las dos clases → tenemos que hacer uso de un diccionario ‘\texttt{map}’.
\newpage
\subsection{Implementación de la relación:}

\begin{center}
	\begin{lstlisting}[frame=single]
class Asignatura;
class Persona{
  public:
    typedef std::map<Asignatura*,int> Bs;
    void setA(Asignatura&, int) noexcept;
    const Bs& getB()const noexcept;
  private:
    Bs bs_;
};

class Asignatura{
  public:
    typedef std::map<Persona*,int> As;
    void setB(Persona&, int)noexcept;
    const As& getA()const noexcept;
  private:
    As as_;
};
/*----------Implementacion de los metodos----------*/
void Persona::setA(Asignatura& a, int atributo) noexcept{
  //Sin permitir que se modifique la clave si existe en el diccionario.
  bs_.insert(std::make_pair(&a,atributo));

  //Permite modificar la clave si ya existe en el diccionario.
  //bs_[&a]=atributo;
}

const Persona::Bs& Persona::getB()const noexcept{
  return bs_; 
}

void Asignatura::setB(Persona& p, int atributo)noexcept{
  //No permite que se modifique la clave si existe en el diccionario.
  as_.insert(std::make_pair(&p,atributo));

  //Permite modificar la clave si ya existe en el diccionario.
  //as_[&p]=atributo;
}

const Asignatura::As& Asignatura::getA()const noexcept {
  return as_;
}
\end{lstlisting}
\end{center}